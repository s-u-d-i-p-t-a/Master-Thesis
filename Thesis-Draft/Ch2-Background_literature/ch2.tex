\documentclass[../report.tex]{subfiles}
\begin{document}	
	
\chapter{Optical waveguide theory}
%To understand the working principle of \gls{pr} it is important to look into the basic concepts of waveguides and the mathematics behind the propagation of \gls{em} waves. Maxwell combined the electric and magnetic fields in a wave equation for a homogeneous medium. Moreover, to understand \gls{pr} in waveguides, it also necessary to look into polarization of light and its formal representation using Jones Calculus. Additionally, Poincaré sphere and Stoke's parameter are required for a representation of \gls{sop}. Apart form these, different \gls{fom} parameters are used to describe transmission parameters in a waveguide.   
		
	\section{Maxwell's equations}
%A wave is an oscillation accompanied by a transfer of energy that travels through medium (space or mass). Waves transfer both energy and momentum, without transferring any mass. \gls{em} radiation is the radiant energy released by varying \gls{em} field in the form of \gls{em} waves. A light wave is \gls{em} radiation at very high frequency. The frequency of visible light falls in between IR and UV \gls{em} waves.
%\begin{figure}[h]
%	\centering
%	\includegraphics[width=0.75\textwidth]{2-light-spectrum}
%	\caption{The EM wave spectrum}
%	\label{fig:2_light_spectrum}
%\end{figure}
%James Clerk Maxwell discovered that he could combine four simple equations, which had been previously discovered, along with a slight modification to describe self-propagating waves of oscillating electric and magnetic fields \cite{waveparticle_2016}. The understanding of propagating light waves using Maxwell's equations in a dielectric medium, is the key to the construction of optical waveguides. Maxwell’s equations relate the electric field $E$ (V/m), magnetic field $H$ (A/m), charge density $\rho$ ($\chem{C/m^3}$), and current density $J$ ($\chem{A/cm^2}$).
%\begin{itemize}	
%	\item \textbf{Maxwell's first equation (Gauss' Law)}: The net electric flux through any closed surface is equal to $\frac{1}{\epsilon_m}$ times the charge density within that closed surface,
%	\begin{equation}\label{eq:max1_1}
%	\nabla \cdot \vec{E} = \frac{\rho}{\epsilon_m},	
%	\end{equation}
%	where $\epsilon_m$ the permittivity of the medium, and the del operator, $\nabla$, is given by:
%	\begin{equation}\label{eq:max1_2}
%	\nabla = \left(\frac{\partial i}{\partial x},\frac{\partial j}{\partial y},\frac{\partial k}{\partial z}\right)
%	\end{equation}
%	where i, j and k are unit vectors in the x, y and z directions respectively.
	
%	\item \textbf{Maxwell's second equation (Gauss' Law for magnetic field)}: The net magnetic flux through a closed surface is always zero, since magnetic monopoles do not exist.
%	\begin{equation}\label{eq:max1_3}
%	\nabla \cdot \vec{H}= 0	
%	\end{equation}
	
%	\item \textbf{Maxwell's third equation (Faraday's law)}: Induced electric field around a closed path is equal to the negative of the time rate of change of magnetic flux enclosed by the path.
%	\begin{equation}\label{eq:max1_4}
%	\nabla\times \vec{E} = -\mu_m\frac{\partial \vec{H}}{\partial t}
%	\end{equation}
%	where $\mu_m$ is the magnetic permeability of the medium.

%	\item \textbf{Maxwell's fourth equation (Modification of Ampere's law)}:  The fourth equation states that magnetic fields can be generated in two ways: by electric current (this was the original “Ampere's law”) and by changing electric fields (this was “Maxwell's addition”) 
%	\begin{equation}\label{eq:max1_5}
%	\nabla\times \vec{H} =  J + \epsilon_m\frac{\partial \vec{E}}{\partial t}	
%	\end{equation}
%	where $\epsilon_m$ is the electric permittivity of the medium.	
%\end{itemize}


\end{document}
